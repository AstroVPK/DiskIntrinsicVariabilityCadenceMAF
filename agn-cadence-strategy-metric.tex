\documentclass[11pt]{article}
\usepackage{graphicx}
\usepackage[top=1.0in,margin=1.0in]{geometry}

\title{\textbf{LSST Cadence Optimization for AGN Variability Science}}
\author{Jackeline Moreno}
\date{July 2016}

 \begin{document}
 \maketitle
    


\section{AGN Lightcurve Simulation and MAF Metric}

The project described here will evaluate how accurately proposed LSST OpSims will perform for AGN optical variability science goals.  Significant continuum variability (over 5-10 \% level) is observed for these sources on timescales from less than a day to several years.  We propose to implement the cadence analysis discussed below into the Metric Analysis Framework.     

Lightcurves will be simulated for AGN across numerous fields in the projected LSST sky footprint including deep drilling fields.  The frequency of visits for all bands in randomly selected fields will be used to generate mock lightcurves for each band by sampling from an original mock realization at a given RA and DEC.  To resemble distant quasars, the complete test ensemble of AGN targets will be simulated as 20th magnitude sources.  The brightness observed in each band will be modelled as differing by either an addition or multiplication of a noise factor.     

To concretely evaluate if regular sampling is necessary, or even significantly advantageous over an irregular pattern that captures both high frequency and low frequency variability,  we propose the following metrics to evaluate a particular observation strategy. 

1. How accurately do we recover the \textbf{amplitude} of variability after estimating observational uncertainty? 

This metric will gauge if we can detect variability in the first place with a given sampling pattern.     


2. Does the OpSim recover the original \textbf{complexity} of the lightcurve variability sufficiently well?
 The "complexity" is statistically parametrized as the order $p$ of the stochastic polynomial that best fits the mock lightcurve. For what fraction of our simulated AGN survey is this metric recovered accurately?
   
   Our mock AGN lightcurves are simulated as C-ARMA processes meaning fluctuations in flux values are driven by random Gaussian noise and dissipate on e-folding times of various strengths.  More complex variability structure in a mock lightcurve can be simulated by increasing the order (number of AR terms) of the C-ARMA polynomial.  This may be developed as a binomial metric (is the estimated $p$ order correct or not), but we will look for any consistent bias that may overestimate or underestimate the complexity $p$ for the entire ensemble of mock lightcurves.  
   
3. Can we recover \textbf{timescales} accurately? This metric will determine how reliably we can make scientific physical inferences about the nature of the observed variability.  The mock lightcurves are generated by directly defining timescales that set the roots of the stochastic polynomial describing the simulated lightcurve using the software library KALI (Kasliwal et al. 2016 in prep, Github:AstroVPK). 

\end{document}